%%%%%%%%%%%%%%%%%%%%%%%%%%%
% Journal Article
% LaTeX Template
% Version 2.0 (February 7, 2023)
%
% This template originates from:
% https://www.LaTeXTemplates.com
%
% Author:
% Vel (vel@latextemplates.com)
%
% License:
% CC BY-NC-SA 4.0 (https://creativecommons.org/licenses/by-nc-sa/4.0/)
%
% NOTE: The bibliography needs to be compiled using the biber engine.
%
%%%%%%%%%%%%%%%%%%%%%%%%%%%%%%%%%%%%%%%%%

%----------------------------------------------------------------------------------------
%	PACKAGES AND OTHER DOCUMENT CONFIGURATIONS
%----------------------------------------------------------------------------------------

\documentclass[
	a4paper, % Paper size, use either a4paper or letterpaper
	11pt, % Default font size, can also use 11pt or 12pt, although this is not recommended
	unnumberedsections, % Comment to enable section numbering
	twoside, % Two side traditional mode where headers and footers change between odd and even pages, comment this option to make them fixed
]{LTJournalArticle}

\addbibresource{sample.bib} % BibLaTeX bibliography file

\runninghead{Building Blocks of a Microkernel} % A shortened article title to appear in the running head, leave this command empty for no running head

\footertext{\textit{Undergraduate Project Report\,|\,CS396A}} % Text to appear in the footer, leave this command empty for no footer text

\setcounter{page}{1} % The page number of the first page, set this to a higher number if the article is to be part of an issue or larger work

%----------------------------------------------------------------------------------------
%	TITLE SECTION
%----------------------------------------------------------------------------------------

\title{Building Blocks of a Microkernel: \\ Lessons from seL4} % Article title, use manual lines breaks (\\) to beautify the layout

% Authors are listed in a comma-separated list with superscript numbers indicating affiliations
% \thanks{} is used for any text that should be placed in a footnote on the first page, such as the corresponding author's email, journal acceptance dates, a copyright/license notice, keywords, etc
\author{%
	\href{https://1-Harshit.github.io/}{Harshit Raj}\thanks{Supervised by: \href{https://www.cse.iitk.ac.in/users/deba/}{Prof. Debadatta Mishra}\\ \textbf{Submitted:} Apr 25, 2023}
}

% Affiliations are output in the \date{} command
\date{\footnotesize Department of Computer Science and Engineering \\ Indian Institute of Technology Kanpur}

% Full-width abstract
\renewcommand{\maketitlehookd}{%
	\begin{abstract}
		\noindent In this report, we provide an analysis of the seL4 microkernel, exploring its building blocks and implementation. The seL4 microkernel is designed to minimize the Trusted Computing Base (TCB) and provide a secure system-building base by moving most kernel functionality to non-privileged user mode. This design enables the microkernel to be formally verified, providing an extra level of assurance in its security and reliability. We examine the seL4 kernel codebase and provide an overview of the kernel objects, components, and capabilities. Additionally, we present a design for building a very minimal microkernel. Overall, this report provides a detailed understanding of a microkernel and its building blocks.
	\end{abstract}
}

%----------------------------------------------------------------------------------------

\begin{document}

\maketitle % Output the title section

%----------------------------------------------------------------------------------------
%	ARTICLE CONTENTS
%----------------------------------------------------------------------------------------

\section{Introduction}

Kernel is a piece of code that serves as the core of an operating system. It manages the system's resources and provides a layer of abstraction between the hardware and software. The kernel is responsible for handling tasks such as memory management, process scheduling, device drivers, and input/output (I/O) operations. All these operations take place in a privileged mode, also know as kernel mode. On the other hand, programs operate in user mode which doesn't allow them to directly access hardware resources. Instead, they need to utilize the operating system to access these resources via different abstractions and interfaces.

Over the years this privileged code has increased with addition of newer technologies and this is as dangerous as useful. If any malitious access is allowed here that could compromise the entire system. This has occured on many mainstream system. Linux kernel comprises of 20 Million lines of source code, it is estimated that it contains tens of thousands of bugs \cite{Biggs:2018}. This idea is captured by saying that Linux has a large trusted computing base (TCB), which is defined as the subset of the overall system that must be trusted to operate correctly for the system to be secure~\cite{Heiser:2020}.

The idea behind a microkernel design is to drastically reduce the TCB to have a secure base to build a system. In a well-designed microkernel, such as seL4, it is of the order of ten thousand lines of source code. As seen in Figure~\ref{fig:microsemantic} the monolithic kernel structure provides every functionality from kernel mode, i.e.\ privileged mode making the TCB large. On the other hand, the microkernel structure provides only the bare minimum functionality from kernel mode and the rest of the functionality is provided by user mode programs. This makes the TCB small and secure. However the downside of this approach is that it requires a lot of user mode programs to be written to provide the functionality and this incurs a performance penalty.

In this report we will be looking at the building blocks of a microkernel and how they are implemented in seL4. We will also be looking at the a design to build a teching microkernel using gem5 simulator~\cite{gem5}. This report aims to explore the concept of microkernels and their significance in designing secure operating systems. A microkernel design addresses this problem by minimizing the TCB to only include the essential functions of the kernel. 

\begin{figure*} % Double column using figure`*`
	\includegraphics[width=\linewidth]{mono-vs-micro.png}
	\caption{To left is a monolithic structure and microkernel structure on right~\cite{Heiser:2020}}
	\label{fig:microsemantic}
\end{figure*}

\newpage

\section{Microkernel Design}

The primary objective of Microkernels is to offer a small and secure kernel that is critical for security. The kernel provides only the essential functionality in kernel mode, and the remaining functionality is provided by user mode programs. This approach helps keep the Trusted Computing Base (TCB) small and secure, which reduces the attack surface. A well-designed microkernel, such as seL4, is typically only about ten thousand lines of source code, in contrast to the Linux kernel, which is approximately 20 million lines of code. This significant difference in size results in a much smaller TCB in a microkernel design, as shown in Figure~\ref{fig:microsemantic}. Let us now consider a microkernel design principle.

\subsection{Minimality \& Generality}

The principles of minimality and Inter-Process Communication (IPC) performance were the primary drivers of Liedtke's designs. He firmly believed that minimality enhances IPC performance. To express this idea, he introduced the microkernel \textit{minimality principle} as follows:~\cite{Liedtke:1995}

\begin{quote}
\textit{A concept is tolerated inside the µ-kernel only if moving it outside the kernel, i.e. permitting competing implementations, would prevent the implementation of the system's required functionality}
\end{quote}

L4 had an implicit driver for the design of kernel mechanisms, which was generality: the aim was to create a foundation on which various systems could be built, almost anything that could run on a processor powerful enough to provide protection.

However, none of the designers of L4 kernels to date claim that they have developed a ``pure'' microkernel that strictly adheres to the minimality principle. For instance, all of them have a scheduler in the kernel, which implements a specific scheduling policy, usually a hard-priority round-robin. To date, no one has been able to come up with a general in-kernel scheduler or a viable mechanism that delegates all scheduling policy to the user level without imposing a significant overhead.

In conclusion, microkernels maintain minimality as a crucial design principle and generality as the overall goal.~\cite{Heiser:2016}

\subsection{Building Blocks}

In this section, we will briefly examine the various building blocks that make up a microkernel.

\subsubsection{Components}

Unlike traditional operating systems, microkernels do not provide any typical OS services. Instead, microkernel designs use \textbf{user-mode} programs called components to provide necessary system services. For example, a file system component provides file system services. These components are loaded into the kernel and run in user mode. The kernel provides the Inter-Process Communication (IPC) mechanism for these components to communicate with each other. Even the \textbf{user application} is a component in this design, as depicted in Figure~\ref{fig:microsemantic}.

\subsubsection{Kernel Objects}

Microkernels also include a limited set of well-defined core kernel objects. These objects run in privileged mode and are responsible for providing basic services, including an entry point for interrupt handling, IPC, basic virtual memory management, thread management, and scheduling. These are the only functionalities provided by the kernel. The rest of the services are provided by the components, as shown in Figure~\ref{fig:microsemantic}.

\subsubsection{Booting System}
As the components and kernel objects are complex and involved, they need to be configured before the system boots up and sent to appropriate locations in memory. This is the responsibility of the build system, which configures the components and kernel objects and sets metadata for the components.


\subsubsection*{} We will discuss each of these building blocks in more detail later.

%------------------------------------------------

% \section{Results}

% \begin{table} % Single column table
% 	\caption{Example single column table.}
% 	\centering
% 	\begin{tabular}{l l r}
% 		\toprule
% 		\multicolumn{2}{c}{Location} \\
% 		\cmidrule(r){1-2}
% 		East Distance & West Distance & Count \\
% 		\midrule
% 		100km & 200km & 422 \\
% 		350km & 1000km & 1833 \\
% 		600km & 1200km & 890 \\
% 		\bottomrule
% 	\end{tabular}
% 	\label{tab:distcounts}
% \end{table}

% Referencing a table using its label: Table\ref{tab:distcounts}.

% \begin{table*} % Full width table (notice the starred environment)
% 	\caption{Example two column table with fixed-width columns.}
% 	\centering % Horizontally center the table
% 	\begin{tabular}{L{0.2\linewidth} L{0.2\linewidth} R{0.15\linewidth}} % Manually specify column alignments with L{}, R{} or C{} and widths as a fixed amount, usually as a proportion of \linewidth
% 		\toprule
% 		\multicolumn{2}{c}{Location} \\
% 		\cmidrule(r){1-2}
% 		East Distance & West Distance & Count \\
% 		\midrule
% 		100km & 200km & 422 \\
% 		350km & 1000km & 1833 \\
% 		600km & 1200km & 890 \\
% 		\bottomrule
% 	\end{tabular}
% \end{table*}

\section{Components}

Components are user-mode programs that provide system services and also function as application programs. The kernel provides an Inter-Process Communication (IPC) mechanism for these components to communicate with each other, if permitted.

\subsection{Nature of Components:}
\begin{itemize}
\item Components are user-mode programs and do not possess special privileges. All logic is implemented in user mode, and components do not have access to kernel mode.
\item Components are independent of each other. If one component fails or is compromised, it does not affect the other components. The only impact is on the service provided by the failed component.
\item Components communicate with each other using the IPC mechanism provided by the kernel.
\item Components must be granted proper access rights at build time to access kernel objects, such as invoking IPC or accessing memory.
\item Components are configurable, allowing them to provide different services, interrupt handling, etc., at build time.
\item Components are reusable, meaning they can be used in different systems.
\end{itemize}

\subsection{Representation of Components}
Components are represented as executables using Thread Control Blocks (TCBs). TCBs are data structures that contain information about the component and are used by the kernel to manage the component's execution. TCBs store information needed for context switching, scheduling, priority, IPC buffer, IPC call, capabilities, etc.

\subsubsection{Data in TCB}
The TCB contains various information, including:
\begin{itemize}
\item Architecture-specific TCB state, such as the program counter and stack pointer.
\item Thread state: a three-word data structure indicating the current state of the component, which can be Inactive, Running, Restart, Blocked on Receive, Blocked on Send, Blocked on Reply, Blocked on Notification, or Idle Thread State.
\item Current fault of the component, i.e., the fault that caused the component to be blocked.
\item Maximum controlled priority of the component.
\item Scheduling context of the component, which contains scheduling parameters such as core and timeslice.
\item Scheduling and notification queues.
\end{itemize}

\subsection{Capabilities}
Capabilities are access rights granted to components to access kernel objects. These access rights are determined at build time and are represented as a data structure in the kernel. Capabilities are configurable at build time and are used to invoke IPC, access memory, etc. Capabilities are essential for implementing the principle of least privilege and determine which component can access which interrupt.


\section{Kernel Objects}

Kernel objects are fundamental components of the microkernel that provide essential services such as interrupt handling, inter-process communication (IPC), basic virtual memory management, thread management, and scheduling. These are the only functionalities that the kernel provides directly. All other services are provided by components that incur an IPC overhead. The kernel objects are implemented in privileged mode. We have already seen one kernel object, the TCB, which is used to manage the execution of components. In this section, we will discuss the other kernel objects, namely system calls, IPC, virtual memory management, and interrupt handling.

\subsection{System Calls}

System calls are the means by which a user-level application interacts with the kernel. In a monolithic kernel, system calls are implemented as a set of functions that are called by the application. Each system call serves a specific purpose, such as opening a file, reading from a file, etc. In a microkernel, all these services are provided by components, and thus, only a limited number of system calls are needed.

\subsubsection{Syscalls in seL4} and are needed in a microkernel are as follows:
\begin{itemize}
	\item Call: This system call is used to invoke a function in a component. The component is identified by its TCB. The function to be invoked is identified by its index in the component's function table. The arguments to the function are passed in the IPC buffer.
	\item ReplyRecv: This system call is used to send a message to a component and wait for a reply. The component is identified by its TCB. The message is passed in the IPC buffer. The reply is also passed in the IPC buffer.
	\item Send: This system call is used to send a message to a component. The component is identified by its TCB. The message is passed in the IPC buffer.
	\item NBSend: This system call is used to send a message to a component. The component is identified by its TCB. The message is passed in the IPC buffer. This system call does not block the caller.
	\item Recv: This system call is used to receive a message from a component. The component is identified by its TCB. The message is passed in the IPC buffer.
	\item NBRecv: This system call is used to receive a message from a component. The component is identified by its TCB. The message is passed in the IPC buffer. This system call does not block the caller.
	\item Reply: This system call is used to send a reply to a component. The component is identified by its TCB. The reply is passed in the IPC buffer.
	\item Yield: This system call is used to yield the CPU to another component.
	\item A microkernel may also need some thread management system calls, such as create thread, delete thread, etc.
	\item Syscalls are also implemented to manage virtual memory, such as map, unmap, etc. But a very elementary virtual memory management is implemented in the kernel. The rest of the virtual memory management is implemented in the components, accessed by a IPC call.
\end{itemize}

\subsection{Inter Process Communication}
Interporcess communication is possibily the most crutial and most discussed topic in microkernel as practicaly everything is an IPC. IPC is implemeted as a Syscall in seL4. Components must be granted proper access rights at build time to invoke IPC to propoer component. The IPC mechanism in seL4 is based on the follwing: IPC buffer, IPC endpoint, and IPC call.

\subsubsection{IPC Buffer}
When a thread wants to send a message to another thread via IPC, it first needs to prepare a message in a buffer. The buffer is allocated from the thread's own memory space, and the message is copied into it. The sender then makes a system call to the kernel to initiate the IPC. The size of the buffer is typically small, which limits the size of messages that can be sent via IPC. However, this design choice makes the IPC mechanism more efficient, as there is no need for dynamic allocation of memory. The Kernel provides support for secure IPC communication by providing capabilities that allow only authorized threads to access a specific buffer.

Alternatively, the messages can be stored in a global buffer shared by all threads. This is the approach taken by the L4 microkernel. In this case, the kernel must ensure that the messages are not overwritten by other threads.  This approach is more efficient than the previous one. However, it is less secure, as any thread can access the buffer.


\subsubsection{IPC Endpoint}
IPC endpoint is a kernel object that represents a communication channel between two threads. It is the basic primitive used for inter-process communication (IPC) in seL4. The endpoint is used to send and receive messages between the threads. When a thread creates an endpoint, it gets a capability that allows it to access the endpoint. This capability can be used to send and receive messages through the endpoint. When a message is sent, the kernel copies the message into the endpoint's buffer and delivers it to the destination thread. When a message is received, the kernel copies the message from the endpoint's buffer into the receiver's buffer.

\subsubsection{IPC Call}
IPC call is implemented as a system call. When a thread wants to initiate an IPC communication with another thread, it makes an IPC system call to the seL4 kernel. The IPC system call takes several arguments, including the capability to the endpoint, the message to be sent, and the length of the message. The kernel provides several different IPC system calls with different semantics, such as blocking and non-blocking variants, as well as variants that allow the calling thread to wait for multiple endpoints at once.

\subsection{Virtual Memory Management}
In the seL4 microkernel, virtual memory management is not provided beyond kernel primitives for manipulating hardware paging structures. Users must provide services for creating intermediate paging structures, as well as for mapping and unmapping pages.

Although users are free to define their own address space layout, there is one restriction: the seL4 kernel claims the high part of the virtual memory range, typically 0xe0000000 and above on most 32-bit platforms.

During the boot process, the seL4 kernel initializes the root task with a top-level hardware virtual memory object, referred to as a VSpace. Once all of the intermediate paging structures have been mapped for a specific virtual address range, physical frames can be mapped into that range by invoking the frame capability.

Overall, the seL4 microkernel provides basic primitives for hardware-level virtual memory management, while leaving higher-level management to user-level services. This allows users to define their own virtual memory layout and implement customized management schemes, while retaining the security and performance benefits of the seL4 kernel.
\subsection{Interrupts}
In the seL4 microkernel, the root task is initially provided with a single capability that grants access to all interrupt (irq) numbers in the system. Components can register themselves to receive interrupts by communicating with the seL4 kernel, which manages the interrupt handling mechanism. When an interrupt occurs, the kernel sends an IPC message to the registered component. The component can then handle the interrupt and send a reply back to the kernel.

\section{Boot System}

Due to the complexity and interdependence of the components and kernel objects in the seL4 microkernel, they must be properly configured and placed in memory before the system can boot up. This task falls under the responsibility of the boot system, which is responsible for configuring the components and kernel objects, as well as setting metadata for the components.

The boot system ensures that all necessary components and kernel objects are properly linked and configured, and that the resulting binary image is compatible with the underlying hardware. In the microkernel, endpoints, capabilities, and interrupt handlers must be set up for all components while booting up. These components are responsible for handling various system services, and their proper initialization is critical for system stability and security. During boot seL4 also creates root server and idle server.

\subsection{Root Server and Idle Server}

The root server in seL4 microkernel provides a centralized resource management service for the entire system. It is responsible for initializing and managing the system's resources, including hardware devices, memory, and other system services. In addition to resource management, the root server is also responsible for initializing the system, and default interrrupt handling.

The idle server, just like monolithic kernels, is a specialized thead that is responsible for managing system health and reducing power consumption. When no user-level process is scheduled for execution, the idle server remains idle, allowing the system to conserve power and reduce resource consumption.

\section{Essential Blocks}
To set up a functional microkernel, there are a few bare minimum requirements that need to be met. These include:
\begin{itemize}
	\item \textbf{Boot System:} A build system is required to configure the components and kernel objects and set metadata for the components. This step is crucial for ensuring that all necessary components are properly configured and initialized before booting up the system.
	
	\item \textbf{Syscall:} A system call interface is necessary for user-level applications to interact with the kernel. The system call interface provides a set of functions that are called by user-level applications to request kernel services such as IPC.
	
	\item \textbf{Inter Process Communication (IPC):} IPC is essential for enabling components to communicate with each other and for coordinating system services.
	
	\item \textbf{Virtual Memory:} Virtual memory management is required to provide each user-level application with its own isolated address space. This ensures that applications cannot access each other's memory and provides a mechanism for protecting the system from rogue applications. This is needed atleast some basic implementation of same.
	
	\item \textbf{Two or more components:} Finally, to create a functional microkernel system, there must be at least two components that are properly configured and initialized. These components are responsible for handling various system services and is also the application user wants to run on system.

\end{itemize}
	
Each of these requirements plays a critical role in the operation of a microkernel-based system. Without these basic components in place, it would not be possible to create a stable and functional system.

\begin{figure*} % Double column using figure`*`
	\includegraphics[width=\linewidth]{example.png}
	\caption{Example of an microkernel based system}
	\label{fig:example}
\end{figure*}

\section{View of a microkernel System}

Suppose we have an application that requires Disk I/O, File Server, and Console to be implemented in a microkernel architecture. To set up this system, we need the following components:

\begin{itemize}
	\item Disk I/O driver (user mode)
	\item File Server (user mode)
	\item Console (user mode)
	\item Application (user mode)
\end{itemize}

Additionally, we need the following kernel services for these components:
\begin{itemize}
	\item Inter-Process Communication (IPC) (kernel mode)
	\item Virtual Memory (kernel mode)
	\item Thread Management (kernel mode)
\end{itemize}

The pictorial representation of this system is shown in Figure \ref{fig:example}. The application is the user-level process that is running on the system.

\subsection{Control Flow}

When the application wants to read a file from the disk, it populates the IPC buffer with the file name and other parameters then sends an IPC message to the File Server. This IPC call is a blocking call, which means that the application will wait until the File Server replies back. The kernel then schedules the File Server to execute. The File Server receives the IPC and then reads the file from the disk. It sends the file contents back or replies to the application using IPC. The kernel then schedules the application to execute again.

\subsection{Fault Handling}

All components are isolated from one another, which means that if the Console component fails or is compromised, the rest of the system remains unharmed and unaffected. The only thing lost in this scenario is the printing service. This is the advantage of a microkernel architecture - the system is more secure and fault-tolerant since each component runs in its own isolated environment.


\section{Future work and Relevant Links}
\begin{itemize}
	\item First iteration of Microkernel implementation in gem5 will be available at \url{https://git.cse.iitk.ac.in/deba/microgemos}
	\item Kernel code of sel4 and tutorial is available at \url{https://github.com/sel4/sel4}
	\item Futute work includes implementing some more user-level services and testing the system.
	\item \href{https://docs.google.com/presentation/d/1ugVbGRoHd7K8TiUXV4uv9d5jXUgYuYz5uG2ctlLUjuc/edit?usp=sharing}{Google Slides Link} 
\end{itemize}

%----------------------------------------------------------------------------------------
%	 REFERENCES
%----------------------------------------------------------------------------------------

\printbibliography% Output the bibliography

\end{document}

